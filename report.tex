\documentclass{article}
\usepackage[utf8]{inputenc}
\usepackage{enumitem}
\usepackage{graphicx}
\usepackage{color}
\usepackage{rotating}

\title{Next Generation User Interfaces: Report}
\begin{document}

\maketitle

\author{Jonas De Bleser, Nicolas Carraggi, Valentijn Spruyt}
\newline
\centerline{\texttt{\{jdeblese, ncarragg, vspruyt\}@vub.ac.be}}
\newline

\tableofcontents

\section{Problem statement}

\section{Requirements analysis}

\subsection{Usability requirements}

\begin{enumerate}
\item 
    \begin{itemize}[label=$ $]
    \item \textbf{Description:} Checking the events per day should be possible within 2 steps starting from the main view
    \item \textbf{Motivation:} Since this is one of the principal aims of the system, it may not be difficult to do this. The user should be able to go to the next, previous or calendar day in one step, as well as applying a filter using another step.
    \item \textbf{User class(es):} User
    \item \textbf{Measuring concept:} User satisfaction
    \item \textbf{Measuring method:} Task scenario
        \begin{itemize}
        \item \textit{Result:} Amount of steps to perform the task
        \end{itemize}
    \item \textbf{Criteria for judging:}
        \begin{itemize}
        \item \textit{Worst level:} 3 or more steps
        \item \textit{Planned level:} 2 steps - using the calendar as well as the filter
        \item \textit{Best level:} 0 steps - The best possible level is zero because checking the events of the current day requires no additional steps
        \end{itemize}
    \end{itemize}
\end{enumerate}
    


\section{Design and iteractions}
 Nicolas Omer zei:

"interaction" was describing how the potential user would interact with the application. In our case, it is an Android application on a smartphone so it is via the touch screen, simple gesture, he uses tabs to navigate, push buttons, etc... (we explain all the things that have to be known to access every feature of the application)

So in our case that would be the 5 movements you have for the Myo I imagine + the fact that you can use another screen than the one of your computer, etc...

For the "design", we did pretty much the same than in the section "Style guidelines" for the project of the "User Interface Design" course (don't know if you have this one). 

Anyway : we describe how the design process was conducted (how we  came to such layout / display). Why a list view is better for storing your accounts (since you could have many), that we have drop-down menus to select a specific item in a list, etc... 

Don't know if that is what they expect but it's what we have wink-emoticon

\section{Technical report}

Inleidend tekstje

\subsection{Architecture}

Java + javascript + myo

\subsection{Challenges}

In this section we will discuss some of the challenges we encountered during the development of our project.

\subsubsection{The Marionette.js framework}
One of the biggest challenges of this project was the Marionette.js\footnote{http://marionettejs.com} framework itself. Even though only one member of our team had experience with the framework, we all agreed on using it. The motivation for using this framework was that it has a neatly built model and view structure, which made it possible to write clean code that is fairly easy to read. The framework itself is a decent framework, but development went very slow because of the basic knowledge of JavaScript by the other two members. On the other side: Using this framework really upped our JavaScript skills, which will be handy for the future.

TODO: Jonas zet nog wa bij die motivatie as ge wilt over marionette. 

\subsubsection{MYO}
onnauwkeurigheid, calibratie, slechts 1 armband dus 1 iemand per keer kan met myo werken en testen.


\subsubsection{JxBrowser}
responsiveness enzo, events doorsturen,



\section{Evaluation}

\subsection{Self-evaluation}
\subsection{Questionnaire}
\subsection{Evaluation by others}

Zelf gedaan + vragenlijst laten rondgaan en laten gebruiken door ouders?


\end{document}
